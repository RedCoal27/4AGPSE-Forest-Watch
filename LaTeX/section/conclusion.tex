\newpage
%le \phantomsection{} permet de regler un bug sur les sommaire qui ne ramène pas a la bonne page
\phantomsection{}
%section sans numéro
\section*{Conclusion}
%ajouter quand même la section dans le sommaire
\addcontentsline{toc}{section}{Conclusion}

Le transistor bipolaire NPN est un composant électronique clé qui permet de moduler l'amplitude d'un signal électrique en fonction de la tension appliquée à la base. Il est composé de trois couches de matériaux semi-conducteurs : l'émetteur, le collecteur et la base. Il fonctionne selon l'effet Schottky. Il existe deux régimes de fonctionnements utilisées : amplification linéaire et commutation (bloqué ou saturé). Les différents montages élémentaires pour utiliser un transistor bipolaire en mode d'amplification sont le montage en émetteur commun, en base commune et en collecteur commun. Le transistor bipolaire NPN est largement utilisé dans les amplificateurs audio, les oscillateurs, les circuits de commutation, les amplificateurs de puissance, les circuits logiques et bien d'autres applications électroniques.