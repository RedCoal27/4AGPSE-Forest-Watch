\newpage
\subsubsection*{DE SOUSA CARDOSO Raimundo Vitor}
The first part of this project allowed me to learn and develop competencies in the areas in which I worked towards the development of the system. Especially those related to modelling and project planning, as well as the communication part via LoRa protocol.
In the part involving planning, I could learn how to organize and detail a project before its execution. I was able to do different analyses, functional, risk, and process, some of them using the Cappella software. 
In the communication via LoRa protocol part, I could understand how the architecture of this technology works. I applied the knowledge obtained in classes to perform a communication via UART between the microcontroller and the LoRa/GPS HAT, also providing a learning and code development for programming this function.
Finally, by working alongside a fantastic team, I was able to develop my teamwork skills, especially in an environment where I was communicating using a foreign language. I learned a lot by debating ideals, discussing problems, pointing out solutions, and presenting our project to teachers and students.



\begin{table}[!h]
    \centering
    \begin{tabular}{|cc|}
    \hline
    \multicolumn{2}{|c|}{\cellcolor[HTML]{32CB00}Raimundo}           \\ \hline
    \multicolumn{1}{|c|}{LoRa}        & 0 →  Intermediate            \\ \hline
    \multicolumn{1}{|c|}{Arduino/C++} & Beginner → Intermediate      \\ \hline
    \multicolumn{1}{|c|}{Raspbian}    & 0 → Intermediate             \\ \hline
    \multicolumn{1}{|c|}{Teamwork}    & Intermediate → Intermediate+ \\ \hline
    \end{tabular}
\end{table}