%====================== PACKAGES ======================

\usepackage[table,xcdraw]{xcolor}
\usepackage{fancyhdr}
\usepackage{titlesec}

\usepackage[english]{babel}
%pour gérer les positionnement d'images
\usepackage{svg}
\usepackage{float}
\usepackage{amsmath}
\usepackage{tikz,graphicx}
\usepackage[colorinlistoftodos]{todonotes}
\usepackage{url}
%pour les informations sur un document compilé en PDF et les liens externes / internes
\usepackage{hyperref}
%pour la mise en page des tableaux
\usepackage{array}
\usepackage{tabularx}
\usepackage{multirow}
%pour utiliser \floatbarrier
%\usepackage{placeins}
%\usepackage{floatrow}
%espacement entre les lignes
\usepackage{setspace}
%modifier la mise en page de l'abstract
\usepackage{abstract}
%police et mise en page (marges) du document
\usepackage[T1]{fontenc}
\usepackage[top=2cm, bottom=2cm, left=2cm, right=2cm]{geometry}
%Pour les galerie d'images
\usepackage{subfigure}
%custom package
%espacement entre les lignes d'un tableau
\renewcommand{\arraystretch}{1.5}

%régler l'espacement entre les lignes
\newcommand{\HRule}{\rule{\linewidth}{0.5mm}}


% liste des auteurs
\newcommand{\authorlist}[0]{4A GPSE, Darrys ABDELKRIM, Raimundo Vitor DE SOUSA CARDOSO, \\Florian KRASULJA, Alexandre MAILLET}


%barre pieds de page
\renewcommand{\footrulewidth}{0.6pt}


%Création de la commande pour le sommaire
\makeatletter
\newcommand*{\tableofcontentssummary}{%
  \begingroup
    \value{tocdepth}=1\relax
    \@fileswfalse
    \renewcommand*{\contentsname}{Summary}%
    \tableofcontents
  \endgroup
}
\makeatother


%tabulation
\newcommand\tab[1][0.5cm]{\hspace*{#1}}


%luckyLux
\newcommand{\forest}{\textit{Forest Watch}}

%to define the current file directory
\usepackage{currfile}
%to generate lipsum text
\usepackage{lipsum}
%to get lhe last page
\usepackage{lastpage}
%source
\usepackage[backend=biber,style=numeric,citestyle=authoryear]{biblatex}

\usepackage{enumitem}
\usepackage{listings}


\hypersetup{
  citebordercolor=white,
  filebordercolor=white,
  linkbordercolor=white,
}